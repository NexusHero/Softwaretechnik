\documentclass[conference]{IEEEtran}
\IEEEoverridecommandlockouts
% The preceding line is only needed to identify funding in the first footnote. If that is unneeded, please comment it out.
\usepackage[nocompress]{cite}
\usepackage{amsmath,amssymb,amsfonts}
\usepackage{algorithmic}
\usepackage{graphicx}
\usepackage{textcomp}
\usepackage{xcolor}

\usepackage{ngerman}
\usepackage[utf8]{inputenc}
\usepackage{longtable}
\usepackage{float}
%\usepackage[breaklinks]{hyperref}
\usepackage{url}

\def\BibTeX{{\rm B\kern-.05em{\sc i\kern-.025em b}\kern-.08em
		T\kern-.1667em\lower.7ex\hbox{E}\kern-.125emX}}
\begin{document}
	
	\title{Multilinguale Spracherkennung mit Deep Neural
		Networks\\
		{\footnotesize \textsuperscript{*}Note: Sub-titles are not captured in Xplore and
			should not be used}
		\thanks{Identify applicable funding agency here. If none, delete this.}
	}
	
	\author{\IEEEauthorblockN{1\textsuperscript{st} Björn Beha}
		\IEEEauthorblockA{\textit{dept. name of organization (of Aff.)} \\
			\textit{name of organization (of Aff.)}\\
			City, Country \\
			email address}
		\and
		\IEEEauthorblockN{2\textsuperscript{nd} Philipp Ginter}
		\IEEEauthorblockA{\textit{dept. name of organization (of Aff.)} \\
			\textit{name of organization (of Aff.)}\\
			City, Country \\
			email address}
		\and
		\IEEEauthorblockN{3\textsuperscript{rd} Suhay Sevinc}
		\IEEEauthorblockA{\textit{dept. name of organization (of Aff.)} \\
			\textit{name of organization (of Aff.)}\\
			City, Country \\
			email address}
	}
	
	\maketitle
	
	\begin{abstract}
	Dieser Artikel beschäftigt sich mit maschinellem Lernen im Bereich der multilingualen Spracherkennung. Die Arbeit setzt sich mit der Forschungsfrage auseinander, welches Verfahren genutzt wird und wie diese mit der Erkennung mehrerer Sprachen arbeitet. Vor allem Sprachen, die wenig Trainingsmaterial bieten stellen hier eine Herausforderung dar. Der Artikel beschreibt, wie die Sprache identifiziert und erkannt wird und wie sich ein entsprechendes System trainieren lässt. 
	\end{abstract}
	
	\begin{IEEEkeywords}
		Deep Learning, Spracherkennung, LSTM
	\end{IEEEkeywords}
	

	
	\input{Subpages/Title}
	\section{Einleitung}\label{sec:introduction}
\IEEEPARstart{S}{ysteme} zur Spracherkennung finden eine zunehmende Verbreitung und Beliebtheit im alltäglichen Leben. Das Spektrum dieser Anwendungen ist dabei Vielfältig und reicht vom Diktieren von Nachrichten über das Steuern von Geräten bis hin zum Einsatz in Autos. Dabei ist die Qualität der Spracherkennung und die Reaktion des Systems ein entscheidender Faktor, um die Interaktion so natürlich wie möglich zu gestalten \cite{Yu.2014}. Allerdings ergibt sich hier ein Hindernis für mehrsprachige Nutzer. Die natürliche Interaktion wird limitiert, da automatische Spracherkennungssysteme den Anwender auf eine voreingestellte Sprache beschränken. In den meisten herkömmlichen Systemen werden Sprachen sowie Dialekte unabhängig voneinander betrachtet. Es wird für jede Sprache ein separates akustisches Modell trainiert \cite{Gonzalez.2015}. Bei weltweit etwa 7000 gesprochenen Sprachen ist es daher nur konsequent, multilinguale Spracherkennungssysteme zu entwickeln \cite{Gary.2018}. Allerdings erfordert ein solches System einen entsprechenden Satz an markierten Trainingsdaten, um wiederkehrende Muster der Sprache zu erkennen. 
Dieser Umstand sorgt für erhebliche Qualitätsunterschiede zwischen den Sprachen, da nicht alle Sprachen über solche Datensätze verfügen. Um die Knappheit der beschrifteten Trainingsdaten zu kompensieren, wird der Ansatz der geteilten Hidden Layer genutzt \cite{Schultz.1998}. Dieser Ansatz stützt sich auf das Zusammenführen aller Daten, um so eine gemeinsame Nutzung der Phoneme zu gewährleisten. Phoneme stellen dabei eine abstrakte Repräsentation aller Laute einer Sprache dar \cite{Zissman.2001}. Um akustische Modelle für eine Vielzahl von Sprachen zu trainieren, um die Kosten, die beim Training dieser Modelle entstehen zu reduzieren und um neue Anwendungsszenarien zu unterstützen, besteht ein wachsendes Interesse an der Entwicklung mehrsprachiger Spracherkennungssysteme \cite{Yu.2014}.
\\\\
Zu Beginn erfolgt die Erläuterung des grundlegenden Aufbaus eines solchen Systems. Dabei werden die Unterschiede zu einem monolingualen System hervorgehoben. Darauf aufbauend wird erläutert, wie die Identifikation sowie die Erkennung von Sprachen funktioniert. Anschließend werden die Recurrent Neural Networks sowie deren Erweiterung, die Long Short Term Memory-Netzwerke (LSTM) beschrieben, welche heute im Bereich der multilingalen Spracherkennung verwendet werden. Dabei wird beleuchtet, wie dieses funktioniert und welche Vorteile gegenüber anderen tiefen neuronalen Netzen bestehen. Das Training eines solchen Modells folgt im Anschluss mit einer abschließenden Diskussion bezüglich weiterhin bestehender Probleme und zukünftiger Ansätze.

\section{Verwandte Arbeiten}
Dieses Kapitel stellt exemplarisch wichtige Arbeiten vor, welche mit dem Thema des Artikels in Beziehung stehen. Es gibt viele Forschungsarbeiten auf dem Gebiet der mehrsprachigen und sprachübergreifenden Spracherkennung. Der Artikel konzentriert sich allerdings nur auf diejenigen, die Recurrent Neural Networks bzw. LSTM-Netzwerke verwenden. Der Schwerpunkt dieser Arbeit liegt bei dem Untersuchen dieser Verfahren zur Realisierung entsprechender Systeme sowie die sich hier ergebenden Vorteile gegenüber bisheriger Verfahren. Dabei wird kein detaillierter Vergleich verschiedener Modelle aufgeführt. \\ 
Die Abhandlung lehnt an das Werk Automatic Speech Recognition - A Deep Learning Approach von Dong Yu und Li Deng \cite{Yu.2014} an. Die Grundlagen bezüglich automatischer Spracherkennungssysteme, konventioneller Ansätze und Trainingsverfahren sowie die Architektur mehrsprachiger Systeme werden hier beschrieben.
Ein weiteres für diesen Artikel interessantes Werk ist das Buch Sprachverarbeitung von Beat Pfister und Tobias Kaufmann, in welchem Grundlagen und Methoden der Sprachsynthese und Spracherkennung genau beschrieben werden.  
Des Weiteren wird in \cite{2017arXiv170307090T} von Tian et al. ein alternatives LSTM-Netzwerk vorgeschlagen, welches den konventionellen Ansatz übertrifft. Auch in den Arbeiten von Graves et al. {\cite{6638947}} und Sak et al. \cite{2014arXiv1402} werden Recurrent Neural Networks im Bezug auf Spracherkennung untersucht. Dabei wird ebenfalls die Effizienz von LSTM-Netzwerken unter anderem durch Experimente bestätigt. \\Da vor allem Sprachen mit einem geringen Satz an Trainingsdaten zu Problemen bei Spracherkennungssystemen führen, ist das Multitask Learning (MTL) ein wichtiger Ansatz multilingualer Sprachsysteme. In der Arbeit von Lu et al. \cite{multitask} werden daher verschiedene Methoden des MTL vorgeschlagen. In dem Artikel wird belegt, dass das Verfahren den Mangel an Ressourcen kompensieren kann, indem auf gemeinsam erlernte Inhalte zurückgegriffen wird.


\section{Hintergrund}
Ein multilinguales Spracherkennungssystem besteht aus mehreren Komponenten. Abbildung \ref{fig:pipeline} illustriert dabei die Pipeline dieser Module. Die Erkennung der Sprache funktioniert über das Aufnehmen von Schallwellen, die beim Sprechen produziert werden. Diese lassen sich über einen elektroakustischen Wandler (Mikrophon) in ein elektrisches Signal umwandeln. Das elektrische Tonsignal wird daraufhin digitalisiert bzw. in Bits konvertiert (Sampling) und über Vorverarbeitung entsprechend aufbereitet, um es in ein neuronales Netz zu speisen {\cite{beat_tobias}}. Diese Daten lassen sich anschließend in Sequenzen aufteilen, aus denen die benötigten Features extrahiert werden. Abbildung \ref{fig:pipeline} stellt diesen Teilschritt als Spektrogramm dar, welches das gesamte Frequenzspektrum visualisiert. Die Feature-Vektoren müssen hier so gewählt werden, dass die kleinste, effizienteste Menge für die Sprachverarbeitung herausgefiltert wird. Die gewonnenen Features dienen schließlich als Eingabe für die Sprachidentifikation. Die gewonnene Information bezüglich der gesprochenen Sprache kombiniert mit den Features werden als Eingabe für den Decoder genutzt. Unter Zuhilfenahme des akustischen Modells sowie des Sprach- und Lexikalmodells, wird der gesprochene Text analysiert und klassifiziert. Die drei Modelle werden von Bäckström \cite{Tom.2016} wie folgt beschrieben:

\begin{itemize}
    \item \textit{Akustikmodell.} Die Menge an Daten, die das neuronale Netz darüber informiert wie die Zusammenhänge zwischen Phonemen und einem konkreten Audiosignal sind. Die Phoneme können
    sowohl kontextabhängig als auch kontextfrei sein.
    Erlernt wird das Modell durch Audioaufnahmen und den zugehörigen Abschriften – die akustischen Trainingsdaten.
    \item \textit{Lexikalmodell.} Dieses Modell bildet eine Sequenz von Phonemen, die durch das Akustikmodell gewonnenen werden, auf gültige Wörter einer Sprache ab. Hierfür werden textuelle Trainingsdaten eingesetzt.
    \item \textit{Sprachmodell.} Die Wahrscheinlichkeit für einen syntaktisch und semantisch korrekten Satz wird genutzt, um aus einer Reihe von Wörtern, die durch das Lexikalmodell gewonnen werden, gültige Wortsequenzen bzw. Sätze zu bilden.
    Beispielsweise folgt auf das englische Wort 'thank' mit einer hohen Wahrscheinlichkeit das Wort 'you' oder
    'god'. Für das Training dieses Modells werden textuelle Trainingsdaten eingesetzt.
\end{itemize}

Jedes dieser drei Modelle muss separat trainiert werden. Das führt zu einer erhöhten Komplexität, verglichen mit dem Trainieren eines einzelnen, gemeinsamen Modells.
Darum wird in den letzten Jahren vermehrt der Ansatz verfolgt, Ende-zu-Ende-Systeme zu entwickeln. Dabei werden die drei Modelle als ein System trainiert und eingesetzt.
Chan et al. \cite{Chan.2015} und Prabhavalkar et al. \cite{Prabhavalkar.2017} zeigen hierzu verbesserte Ergebnisse, verglichen mit mehreren Systemen welche auf einzelne Modelle bauen.

\begin{figure}[h!]
    \centering
    \includegraphics[width=1\linewidth]{images/pipeline}
    \caption{Pipeline eines Spracherkennungssystems (Eigene Darstellung, in Anlehnung an: \cite{Tom.2016}) }%\cite{??}}
    \label{fig:pipeline}
\end{figure}
	\section{Sprachidentifikation}
\subsection{Überblick}
Systeme zur Sprachidentifikation werden eingesetzt um die Sprache eines Audiosignals zu klassifizieren. Üblicherweise ist dies der erste Schritt in automatischen multilingualen Spracherkennungssystemen.
Ohne die richtige Ausgangssprache, können Ausdrücke und Grammatikregeln nicht erkannt werden können \cite{Bartz.2017}.
Die Einsatzgebiete lassen sich laut \cite{Zissman.2001} in zwei Kategorien einteilen. Zum Einen ist es die Vorverarbeitung für maschinelle Systeme und zum Anderen für menschliche Zuhörer. Ein Beispiel für ersteres könnte ein sprachgesteuertes System – an einem Flughafen - zur Flugauskunft sein. Ohne menschliches Zutun könnte das System zuerst die Sprachidentifikation ausführen und danach mithilfe des korrekten Sprachmodells die gesprochene Sprache erkennen.
Eine Vorverarbeitung für menschliche Zuhörer könnte ein Notrufsystem sein, bei dem zuerst die Sprache des Anrufers erkannt. Anhand dieser wird der Anruf dann an einen Mitarbeiter der für diese Sprache zuständig ist weitergeleitet. Besonders relevant ist solch ein System, da in einem Notfall jede Minute über Leben und Tod entscheiden kann.

\subsection{Wie wird eine Sprache identifiziert?}
Eine Sprache wird von Menschen und Maschinen anhand der Unterschiede zwischen den Sprachen identifiziert werden. \cite{Zissman.2001} nennt hierfür die folgenden Charakteristika:
\begin{itemize}
\item \textit{Phonologie.} Hier wird die Häufigkeit und Verteilung von Phonemen und Phonen betrachtet. Ein Phon ist der tatsächlich produzierte Ton, der beim Sprechen entsteht.
\item \textit{Morphologie.} Sprachen unterscheiden sich in den Wortstämmen, dem Vokabular und der Art, wie Wörter geformt werden.
\item \textit{Syntax.} Sätze haben in unterschiedlichen Sprachen, unterschiedliche Satzstrukturen.
\item \textit{Prosodie.} Temp, Rhythmus, Pausen und Tonhöhen unterscheiden sich von Sprache zu Sprache.
\end{itemize}

\subsection{Architektur}
\cite{Gonzalez.2015} Die Umsetzung der Sprachidentifikation spiegelt sich in der gewählten Architektur eines automatischen Spracherkennungssystems wieder. \cite{Gonzalez.2015} unterscheidet hierbei zwischen drei möglichen Umsetzungen.

[4-6] Die erste ist es, ein universelles Modell zu trainieren, indem alle Sprachen als Eingabe möglich sind. Die Hidden Layer eines neuronalen Netzes teilen sich die Repräsentationen für Phoneme und in den Ausgabeschichten wird die Sprache erkannt.
Vorteile ergeben sich durch die gemeinsame Nutzung von Phonemen. Zwischen den einzelnen Sprachen gibt jeweils gleiche Phoneme, die nicht mehr erneut erlernt werden müssen. Möchte man eine neue Sprache trainieren, so kann man auf die bereits vorhandenen Strukturen aufbauen und diese mitnutzen.
Das gesamte System wird auch nicht mehr so komplex, wie einzelne monolinguale Systeme \cite{Bartz.2017}.

[7-8] Eine weitere Möglichkeit – der Identifikation einer Sprache – ist der Einsatz eines dedizierten Systems. Anhand eines Teilstückes des Eingabesignals wird die Sprache bestimmt. \cite{Niesler.2006} gibt hierfür eine durchschnittliche Dauer von 2,3 Sekunden an, bestätigt jedoch auch, dass mit zunehmender Länge die Genauigkeit zunimmt.
Nach \cite{Niesler.2006} gibt es für ein System zur Sprachidentifikation mehrere Ansätze. Neben dem Einsatz eines Gaussian mixture models (GMM) gibt es auch Systeme die Neuronale Netze einsetzen. Diese beiden können wiederum unterteilt werden in die Erkennung von Wörtern oder Phons.
Ein Nachteil der Spracherkennung anhand eines Teilstückes ist eine erhöhte Latenz. Sollte außerdem die Sprache falsch erkannt worden sein, breitet sich der Fehler aus und führt zu einem möglicherweise falschen Endresultat.

[9] Die dritte und zugleich letzte genannte Möglichkeit, setzt auf mehrere monolinguale Spracherkennungssysteme. Das Eingangssignal wird simultan von mehreren Systemen mit jeweils eigenen Modellen verarbeitet.
Anhand der größten Übereinstimmung mit einer Sprache, wird am Ende dann die passende Sprache ausgewählt. Es werden wiederum die Probleme des vorherigen Ansatzes gelöst und es wird mit einer höheren Sicherheit die richtige Sprache ausgewählt.
Nachteil ist hierbei der erhöhte Rechenaufwand.

In den nachfolgenden Kapiteln gehen wir von dem ersten Ansatz aus, da es hierzu die aktuellsten Forschungen gibt \cite{Gonzalez.2015} <<Wo steht noch mehr dazu drin?>>

\subsection{Ausblick, Fazit, Praktische Umsetzung, Auf was gehen wir ein?}
- Werte: welche Architektur ist am besten
- Auf was gehen wir ein und wieso?
	\section{Multilinguale Spracherkennung}
Die Kernidee der mehrsprachigen Spracherkennung ist bei den verschiedenen Architekturen dieselbe. Die Hidden-Layer des Deep Neural Networks können als ein intelligentes Merkmalsextraktionsmodul betrachtet werden, welches aus mehreren Quellsprachen trainiert wird. Nur die Ausgabeschicht liefert eine direkte Übereinstimmung mit den relevanten Klassen. So lassen sich die Extraktoren für eine Reihe verschiedener Sprachen gemeinsam nutzen. Wie im vorherigen Kapitel bereits erläutert wurde, lässt sich somit besonders das Problem beim Lernen der tiefen neuronalen Netze entgegenwirken. Diese lassen sich sind aufgrund ihrer Parameter und dem sogenannten Backpropagation-Algorithmus langsamer trainieren als andere Modelle. Ein weiterer Vorteil, den dieser Ansatz bietet ist, dass auch mit Sprachen, die nur einen geringen Satz an markierten Trainingsdaten bietet, erlernt werden können, indem Elemente anderer Sprachen übertragen werden. Merkmale, die aus diesen neuronalen Netzen extrahiert werden, lassen sich kombinieren, um so die Erkennungsgenauigkeit zu verbessern [1]. 
Eine gemeinsame Nutzung wird ermöglicht, indem Phoneme gemeinsam genutzt werden. Phoneme sind als kleinste, bedeutungsunterscheidende Einheiten der Lautsprache definiert. Phoneme werden zur Repräsentation der Aussprache genutzt. Um obige Ansätze zu nutzen, müssen Beziehungen zwischen den akustischen Signalen der Sprachen erkannt werden. Jede Sprache besitzt dabei ihre eigenen Charakteristika. In der Sprachübergreifenden Erkennung gibt es einen Satz aus trainierten sowie untrainierten bzw. schlecht trainierten Phonemen, die erkannt werden müssen. Die Töne einer Sprache müssen mit einem ähnlichen bzw. dem ähnlichsten trainierten Ton einer anderen Sprache ersetzt werden. Beispielsweise gibt es den Ton /y/, welcher im Wort ‚süß‘ vorkommt. Wenn ein System nun mit der deutschen Sprache genutzt wird, welches nur in anderen Sprachen trainiert wurde, muss der ähnlichste Sound zu /y/ gefunden werden [5]. So lassen sich phonetische Klassen aus mehreren Sprachen unterscheiden. Ein Transfer des Modells ist trivial. Es wird leidglich eine neue Softmax-Schicht angelegt und trainiert. Die Softmax-Funktion wird zur Klassifikation verwendet [6]. Die Ausgabenkoten dieser Schicht entsprechen dann den Senonen der Zielsprache. Senonen beschreiben lediglich das Betrachten des lautlichen Kontextes der einzelnen Phoneme und stellen gebundene Triphonzustände dar [1]. Die Kontexte können komplex sein [7]. 
Die Softmax-Schicht wird nur auf die entsprechende Sprache trainiert. Weitere Verbesserungen lassen sich erzielen, indem das gesamte Netzwerk zusätzlich auf die neue Sprache abgestimmt wird. Ein solche Architektur ist in Abbildung (…) illustriert. Sie zeigt die gemeinsam genutzten Schichten, die die Merkmale extrahieren sowie die unterschiedlichen Input-Datensätze. Jede Sprache hat ihre eigene Softmax-Ebene. Mit der Softmax-Funktion lassen sich hier die Zustände des akustischen Modelles vorhersagen bzw. entsprechende Wahrscheinlichkeiten schätzen. Wird ein neuer Datensatz in das System gegeben, werden nur die sprachspezifische Schicht sowie die Hidden Layer angepasst. Andere Softmax-Schichten bleiben intakt. Nach dem trainieren der fünf Sprachen ist das System in der Lage diese fünf Sprachen zu erkennen. Die Erweiterung um eine Sprache ist trivial. Kommt eine weitere Sprache hinzu, wird leidglich eine neue Softmax-Ebene an das vorhandene Netzwerk angefügt und trainiert [1]. 

\begin{figure*}[h!]
	\centering
	\includegraphics[width=1.0\linewidth]{images/shared_hidden_layer}
	\caption{Hinzufügen einer neuen Sprache  \cite{GonzalezDominguez.2015}} %Generelle
	\label{fig:topology}
\end{figure*}

Ein Vergleich eines monolingualen Deep Neural Networks und eines multilingualen Deep Neural Networks ist in Tabelle (…) aufgeführt. Das monolinguale Netzwerk wurde hierbei nur mit der entsprechenden Sprache trainiert, während das multilinguale System mit allen vier Sprachen trainiert wurde. Dabei wird die prozentuale Wortfehlerrate (Word error rate, WER) angegeben. Es ist zu erkennen, dass das multilinguale System das monolinguale in allen Sprachen übertrifft. Diese Verbesserung ist dem sprachübergreifenden Wissen zuzuschreiben [1]. 

\begin{table*}[h!]
	\begin{tabular}{lllll}
		& FRA             & DEU           & ESP           & ITA             \\
		Test set size (words) & 40k             & 37k           & 18k           & 31k             \\
		Monolingual DNN WER   & 28.1\%          & 24.0\%        & 30.6\%        & 24.3\%          \\
		Mulitlingual DNN WER  & 27.1\% (-3.6\%) & 22.7 (-5.4\%) & 29.4 (-3.9\%) & 23.5\% (-3.3\%)
	\end{tabular}
	\centering
	\caption{Relative Wortfehlerrate}
	\label{my-label}
\end{table*}

In [2] wurden eine Reihe weiterer Versuche durchgeführt, um die Wirksamkeit eines solchen Systems zu evaluieren. Dabei wurde zwei verschiedene Zielsprachen verwendet. Zum einen das amerikanische Englisch, welches phonetisch nahe an den europäischen Sprachen der Tabelle (…) liegt und Mandarin-Chinesisch, welches weit von den europäischen Sprachen entfernt ist. 
Die tatsächliche Erkennung der Sprache ist dabei trivial. Die Schallwellen, die beim Sprechen produziert werden, lassen sich über einen elektroakustischen Wandler (Mikrophon) in ein elektrisches Signal umwandeln. Dieses elektrische Tonsignal wird daraufhin in Zahlen bzw. Bits konvertiert (sampling) und über Vorverarbeitung entsprechend aufbereitet, um es in ein neuronales Netz zu speisen [4]. Beim genauen Vorhersagen des gesprochenen kommt die Sprachidentifikation ins Spiel, durch welche Wörter ausgeschlossen werden, die ebenfalls in Frage kommen, allerdings zum Wortschatz einer anderen Sprache gehören. Hat ein System z.B. die deutsche Sprache erkannt, wird für das Wort „Hello“ immer „Hallo“ vorhergesagt, da dies naheliegender ist. Es wird hier mit statistischen Modellen gearbeitet, um anzugeben mit welcher Wahrscheinlichkeit welches Wort vorkommt oder aufeinander folgen können. Dabei gibt es verschiedene Lösungsansätze, um das gesprochene vorherzusagen. Oft werden tiefe neuronale Netze in Verbindung mit Hidden Markov-Modellen eingesetzt. Diese hybriden Systeme werden in der Literatur oft untersucht und beschrieben. Ein allerdings leistungsfähigeres Modell bieten die Recurrent Neural Networks. Diese Form von neuronalen Netzen werden heutzutage eingesetzt und erreichen hohe Genauigkeiten [1].


\section{Recurrent Neural Networks}
Im Bereich der Spracherkennung werden heutzutage sogenannte Recurrent Neural Networks eingesetzt, durch welche die Netzwerke ihre Spracherkennungsgenauigkeit erreichen. Das Modell dieser Netze erlaubt gerichtete zyklische Verbindungen zwischen den Neuronen, wodurch es mit einem temporalen Verhalten ausgestattet wird. Recurrent Networks sind somit ideal zum Lernen von Datensequenzen geeignet. Sprache, also kontinuierliche Audiostreams fallen somit ebenfalls in das Anwendungsgebiet dieser Netzwerke. Diese Form von neuronalen Netzen unterscheidet sich grundlegend von dem Feed-Forward-DNN, da es nicht nur basierend auf Eingaben arbeitet, sondern auch auf interne Zustände zurückgreift. Diese internen Zustände speichern die vergangenen Informationen in der zeitlichen Reihenfolge, in welcher diese verarbeitet wurden. Somit ist ein RNN deutlich dynamischer, als ein Deep Neural Network, welches lediglich eine statische Eingabe-Ausgabe-Transformation durchführt. Dabei wird eine Erweiterung des Backpropagation-Algorithmus eingesetzt. Die Backpropagation-Through-Time-Methode sorgt für das Berechnen der Gradienten. Diese werden im Gegensatz zum Standard-Algorithmus über die einzelnen Zeitschritte aufsummiert. In dieser Erweiterung des Backpropagation, welche in Recurren Neural Networks eingesetzt wird, werden lediglich die Parameter einzelnen Zeitschritte zwischen den Ebenen geteilt. In Abbildung (…) ist ein vereinfachtes Modell illustriert [1].

\begin{figure*}[h!]
	\centering
	\includegraphics[width=0.5\linewidth]{images/rnn}
	\caption{Modell des Recurrent Neural Network  \cite{GonzalezDominguez.2015}} %Generelle
	\label{fig:topology}
\end{figure*}

Die Abbildung zeigt eine Folge von Iterationen. Der Input ist in obiger Darstellung x, s bezeichnet den Schritt und E den Hidden State, welcher sich beim Eingeben des Inputs ergibt. Ein Recurrent Network gibt somit nicht nur den Input an die nächste Iteration, sondern Input sowie den resultierenden Zustand E. Somit beeinflussen die vorhergehenden Schritte die folgenden. Dies führt zu einem Problem -dem Verschwinden von Information bzw. dem Vanishing Gradient Problem, welches sich dadurch ergibt, dass RNNs nicht in der Lage sind, auf Informationen zurückzugreifen, die weit in der Vergangenheit liegen.
Wenn eine Datensequenz lange ist und das System versucht die gesagten Worte vorherzusagen, kann es sein, dass der Kontext bereits vergessen wurde und eine inkorrekte Vorhersage stattfindet. Somit kommt eine erweiterte Form des RNNs zum Einsatz. Dieses wird Long-Short-Term-Memory (LSTM) genannt und erzielt enorm gute Ergebnisse in automatischen Spracherkennungssystemen [1][3]. 
Diese Netzwerke sind somit in der Lage anhand des Kontextes zukünftige Wörter vorherzusagen und so ihre Genauigkeit zu erhöhen.  Auch mit verrauschten Aufnahmen oder schlechteren Bedingungen beim Aufnehmen des Gesprochenen kann diese Form von Netzwerken bessere Ergebnisse erzielen. 
Aufgrund dessen wurden LSTM-Netzwerke entwickelt, die zur Lösung des Problems beitragen. Dabei werden Recurrent Neural Networks mit einer Speicherstruktur erweitert, was zur namensgebenden Lang-Kurzzeit-Speicherung führt. Diese erlauben die Erkennung zeitlich ausgedehnter Muster und das Erkennen von Zusammenhängen von zeitlich getrennten Ereignissen. Somit eignen sich die Netzwerke um Zeitreihen zu verarbeiten und vorherzusagen. Sogar, wenn zwischen wichtigen Ereignissen sehr lange Verzögerungen liegen, die eine unbekannte Länge aufweisen. Auch bei der Erlernung geräuschverzerrter und hallender Sprachmerkmale kann dieses Modell genutzt werden [1]. 
Die grundsätzliche Idee dabei ist es über elementweise Multiplikationen den Informationsfluss in dem Netzwerk zu steuern. Es kann als komplexe und intelligente Netzwerkeinheit betrachtet werden, welche Informationen über einen langen Zeitraum speichern kann. Dies wird durch die Gating-Struktur erreicht, die bestimmt, wann die Eingabe signifikant genug ist, um sich daran zu erinnern, wann sie sich die Information weiter merken oder vergessen sollte und wann sie die Information ausgeben sollte. Dies geschieht über verschiedene Gates innerhalb einer LSTM-Zelle (…). Ein Gate ist dabei nichts weiter, als eine Reihe von Multiplikationen bzw. Matrixoperationen [1].
Das System ist somit in der Lage aus dem Kontext heraus genaue Vorhersagen zu treffen, wodurch Spracheerkennung deutlich präziser wird. 
Allerdings ist es selbst heute nicht möglich das Spracherkennungsproblem allgemein zu lösen. Spracherkennungssysteme werden somit nur für bestimmte Anwendungsfälle oder Szenarien konzipiert. Mit einer solchen Spezialisierung auf entsprechende Anwendungsgebiete können zum einen höhere Genauigkeiten erreicht werden und zum anderen wird nicht so viel Rechenleistung und Speicher benötigt [4]. Vor allem bei der multilingualen Spracherkennung besteht die Schwierigkeit Gemeinsamkeiten verschiedener Sprachen zu nutzen, um Sprachen mit wenig Trainingsdaten mit einer ausreichenden Genauigkeit anzubieten. Es gilt die Sprachen zu finden, die zur besten Erkennungsleistung der neuen Sprache führen. Dabei müssen Beziehungen zwischen den Sprachen erkannt werden. Problematisch ist auch, dass gleiche Phoneme je nach Sprecher, Sprache etc. variieren, was dazu führt, dass Phoneme nur im Kontext betrachtet werden (Triphone). 
	\section{Trainingsvorgang}

Der Trainingsvorgang basiert auf ein mehrschichtiges tiefen neuronalen Netzwerks. Das Netzwerk aus Neuronen besteht aus drei Schichten:
\begin{description}
	\item Input-Schicht
	\item Hidden-Schicht 
	\item Output-Schicht
\end{description}
Die Input-Schicht stellt die Eingangsdaten dar, welche als Trainingsmaterial dient. Bei diesen Daten handelt es sich um Sprachaufnahmen. Bei Bedarf können diese Aufnahmen durch Filter vorverarbeitet werden. Anschließend werden die beschrifteten Daten in die Netztopologie eingespeist. Eine Vorklassifizierung der Sprache führt zu einer erhöhten Spracherkennungsrate von mehreren Sprachen, da diese Methode sich für mehrere Klassifiziuerungsklassen eignet. In der Hidden-Schicht geschieht das Training. Hier werden die Phoneme der Sprachen extrahiert und gelernt. Dabei wird die Sigmoid-Funktion als Aktivitätsfunktion eingesetzt (s. Formel \ref{normal}). Diese Funktion beschreibt den Korrelation zwischen Input-Wert und  Aktivitätslevel eines Neurons dar. Zudem wird der Input-Wert auf die X-Achse eingetragen. Auf die Y-Achse wird der zugehörige Aktivitätslevel eingetragen. Der Aktivitätslevel wird durch eine Ausgabefunktion in den Output transformiert, den das Neuron an andere Neuronen weitersendet\cite{Neuronal31:online}. Das Netz wird beginnend von der Input-Schicht bis Output-Schicht vollständig durchlaufen. 
\begin{equation}
sigm(x)=\frac{ 1 }{1+e^{-x}  }
\label{normal}
\end{equation}
Sobald die Output-Schicht erreicht ist, wird das Netz rückwärts durchlaufen. Dieses Verfahren wird auch Gradientenabstiegsverfahren genannt und wird benötigt, um fehlerhafte Kantengewichte herauszufinden und anzupassen. Die Kantengewichte des Netzes werden mit null initialisiert. Die Ableitung der Sigmoid-Funktion wird bei der Korrekturberechnung notwendig (s. Formel \ref{ableitung}). Bei größeren Datenmengen entsteht ein Nachteil, welches sich auf die Wissensausprägung des Netzes auswirkt. Beim rückwärts durchlaufen entsteht ein Wissensverlust \cite{bishop.2006}. Dieser Verlust wird durch das Maxima der Ableitung $sigm(x)'$ repräsentiert. Dieser kann bis zu 25 \% betragen. Der entstehende Verlust würde die Klassifizierungsrate des Trainingsmodels reduzieren, welches in Abbildung \ref{fig:features11.0} dargestellt ist \cite{Kulbear.2017}.
\begin{equation}
sigm(x)'= \frac{ e^{x} }{(e^{x} +1)^2  }
\label{ableitung}
\end{equation}

\begin{figure}[h!]
	\centering
	\includegraphics[width=1.0\linewidth]{images/sigmund}
	\caption{Darstellung der Sigmoid-Funktion und dessen Ableitung \cite{Kulbear.2017}} %Generelle
	\label{fig:features11.0}
\end{figure}
Anstelle der Sigmoid-Funktion als Aktivitätsfunktion wird in den State-Of-the-Art-Tiefes-Lernen-Netzen rectified linear units ($ReLUs$) verwendet (s. Formel \ref{eq:ReLU}). Diese Funktion ist dem menschlichen Neuron am ähnlichsten und bringt zudem eine erhöhte Verarbeitungsgeschwindigkeit mit sich \cite{zeiler.2013}. Die Berechnung der Kantengewichte erfolgt durch Formel \ref{eq:Gewichte}).
\begin{equation}
y_{j} = ReLU(x_{j}) = max(0,x_{j}) 
\label{eq:ReLU}
%\caption{Rectiefied linear Units als Aktivierungsfunktion}
\end{equation}
\begin{equation}
x_{ j } = b_{ j } + \sum{ }{ }{ x_{ ij } * y_{j}}
\label{eq:Gewichte}
\end{equation}
Als Nächstes folgt die Output-Schicht, welches die Eingangsdaten zu den Klassen (Zielsprachen) zuordnet. Diese Schicht ist als Softlayer konfiguriert, welches die Klassen in eine eindimensionale Matrix kategorisiert. Dabei ist die Matrix in dem Zahlenintervall $[0,1]$ normalisiert. Die endgültige Sprachidentifikation geschieht über normalisierte Werte, welches in Abbildung \ref{fig:soft} dargestellt wird. Die Werte können in Wahrscheinlichkeiten ausgedrückt werden, in dem die Matrixwerte mit dem Faktor 100 multipliziert werden \cite{Kulbear.2017}.
\begin{figure}[h!]
	\centering
	\includegraphics[width=0.7\linewidth]{images/softmax}
	\caption{Klassenzuordnung über Wahrscheinlichkeiten in der Softmax-Konfiguration \cite{Kulbear.2017}} %Generelle
	\label{fig:soft}
\end{figure}
 Die Vorhersagen der Output-Schicht geschieht durch die Funktion $p(j)$  (s. Formel \ref{eq:soft}). Dabei steht der Index l für die jeweilige Sprache.
\begin{equation}
p(j)= \frac{ exp(x_{j}) }{\sum_{l}{}{ exp(x_{l})} }
\label{eq:soft}
\end{equation}
Für den vorhin erwähnten Gradientenabstiegsverfahren wird ebenfalls eine Kostenfunktion benötigt. Diese geschieht durch Cross-Entropy-Loss-Funktion. 
\begin{equation}
C= \sum_{l}{}{ t_{j} * log(p_{j})} 
\label{eq:back}
\end{equation}
Diese Funktion misst die Abweichungen der Kantengewichte der Netztopologie und passt diese rückwirkend an. Der Cross-Entropy-Verlust nimmt zu, wenn der vorhergesagte Wert von der tatsächlichen Beschriftung abweicht \cite{MLCheatsheet.2017}. Bei $t_{j}$ handelt es sich um die Klasse, für die der Verlust berechnet wird \cite{GonzalezDominguez.2015}.

\subsection{Netztopologie}
Die Netztopologie beschreibt die Infrastruktur des Netzes. Die Auswahl der Topologie bestimmt die Qualität des Trainingsvorgangs. Eine zu geringe Anzahl der Neuronen führt zu einer niedrigen Spracherkennungsrate. Wiederum eine zu hohe Anzahl würde zu überhöhten Trainingsdauer führen. Aufgrund dessen fallen Topologien von Ansatz zu Ansatz unterschiedlich aus, welche unterschiedliche Spracherkennungsresultate liefern \cite{bishop.2006}. In dieser Arbeit wird der Topologienvorschlag von Gonzales et al. betrachtet. Für die Eingangsdaten werden 40 Filterbanken verwendet. In der Input-Schicht werden 26 Neuronen eingesetzt. Um unerwünschte Latenzzeiten zu vermeiden wird ein asymmetrischer Kontext verwendet. Die Hidden-Schicht beträgt vier Ebenen mit einer Gesamtzahl von 2560 Neuronen. Die Output-Schicht enthält wie bereits erwähnt eine Softmax-Konfiguration, dessen Dimension der Anzahl der Zielsprachen entspricht. Dies ist bei der Erkennung von multilingualen Sprachen eine erforderliche Konfiguration. \cite{GonzalezDominguez.2015}.
 
%\begin{figure*}[h!]
%	\centering
%	\includegraphics[width=1.0\linewidth]{images/Output}
%	\caption{Netztopologie zur  \cite{GonzalezDominguez.2015}} %Generelle
%	\label{fig:topology}
%\end{figure*}
\subsection{Verbesserung des Trainingsverfahrens durch Multitasking learning (MTL)}
 Bei maschinellem Lernen wird der Fokus auf das Optimieren bestimmte Metriken, wie beispielsweise Klassifizierungsgenauigkeit und Trainingsdauer, gesetzt. Das Lernen der einzelnen Sprachen läuft sequenziell ab. Hier setzt das Multitasking Learning (MTL) ein. Es werden mehrere Lernaufgaben gleichzeitig erledigt statt sequentiell, um das Trainingsvefahren effizienter zu gestalten. Das führt zu einer verbesserten Lerneffizienz und Vorhersagegenauigkeit. Der Schlüssel zur erfolgreichen Anwendung von MTL besteht darin, dass die Aufgaben miteinander verknüpft werden. Dies bedeutet nicht, dass die Aufgaben ähnlich sein müssen. Stattdessen bedeutet es, dass Aufgaben auf verschiedene Ebenen abstrahiert und geteilt werden. Dabei kann das Wissen zwischen Aufgaben  übertragen werden, welches die Trainingsdauer deutlich verkürzt. MTL ist vor allem nützlich, wenn die Größe des Trainingssatzes im Vergleich zur Modellgröße klein ist. Dabei wird grundsätzlich zwei Arten von MTL unterschieden: Hard parameter sharing und soft parameter sharing. \\ \\ Hard parameter sharing stellt das meist genutzte Art dar\cite{Ruder.2017}. Es wird normalerweise auf die Hidden-Schicht angewendet, indem die Aufgaben gemeinsam gelernt werden, während die spezifischen Aufgaben separat gelernt werden. Dies wird in Abbildung \ref{fig:hard} dargestellt. Dieser Ansatz reduziert das Risiko von overfitting erheblich. Je mehr Aufgaben gleichzeitig gelernt wird, desto mehr muss das Modell eine Repräsentation finden, die alle Aufgaben erfassen muss. Dadurch ist die Chance auf overfitting deutlich geringer \cite{Ruder.2017} \cite{Lu_multitasklearning}.
  \begin{figure}[h!]
 	\centering
 	\includegraphics[width=0.8\linewidth]{images/hard}
 	\caption{Hard parameter sharing auf die Hidden-Schicht angewendet \cite{Kulbear.2017}.} %Generelle
 	\label{fig:hard}
 \end{figure}


	\section{Diskussion und Ausblick}
Die menschliche Sprache ist der natürlichste Weg etwas zu kommunizieren, so ist es nicht wunderlich, dass das Interesse an dem Deep Learning-Ansatz für Spracherkennung und damit verbundene Anwendungen steigt. In dieser Arbeit wurde das Gegenstück zu den konventionellen, stochastischen Modellen beleuchtet- die Recurrent Neural Networks mit der Erweiterung der LSTM-Struktur. Dabei geht hervor, dass bei diesen Netzen die einzelnen Neuronen nicht isoliert betrachtet werden können. Vielmehr hängt deren Zustand und Aktivierung von den Aktivitäten anderer Neuronen ab. Vorhergehenden Ereignissen beeinflussen den Zustand. Durch die dynamische Rekursion wird ein Gedächtnis geschaffen, mit welchem sich die Netze an vergangene Zustände erinnern können und aufgrund dieser Erfahrungen genauere Vorhersagen treffen. Vor allem bei Datensequenzen ist dies sehr hilfreich und da die meschliche Sprache lediglich eine Sequenz von Tönen ist, eignet sich diese Form von Netzwerken ideal. Somit sind diese Netzwerke robust gegenüber Störgeräuschen. Es ging hervor, dass mit der multilingualen Spracherkennung bessere Ergebnisse erzielt werden, als mit der monolingualen Erkennung. Dies ist auf das gemeinsame Nutzen der Phoneme zurückzuführen. Heutige Genauigkeiten beim Erkennen von Sprachen erreichen die Wort-Fehler-Rate eines Menschen. Somit ist das reine Verstehen nicht das Hauptsächliche Problem. Die Autoren sind der Meinung, dass eine natürliche Interaktion mit einem Spracherkennungssystem schwierig bleibt, solange das System keine Kenntnisse über die Welt hat. Beispielsweise klingen im Deutschen die Worte Meer und mehr gleich, haben jedoch nichts gemeinsam. Diese Homophone lassen sich zwar verstehen, das Spracherkennungssystem erkennt allerdings nicht den Kontext und es könnte zu einer inkorrekten Vorhersage führen. Es bestehen auch weitere, zahlreiche limitierende Faktoren. Wie in der Arbeit beschrieben zählen hierzu vor allem auch Sprachen, die keine ausreichenden Ressourcen bieten. Das Mapping von Wörtern und Sequenzen aus Phonemen braucht Linguistikexperten und stellt eine Herausforderung dar. Schließlich müssen sämtliche Phoneme der Sprache identifiziert werden. Auch der Stil beim Sprechen verändert sich ständig und ist nie gleich zwischen unterschiedlichen Sprechern. Gesprochene Wörter beeinflussen die Betonung der nächsten Worte.   
Die Zukunft im Bereich des maschinellen Lernens bleibt spannend. Wir sind überzeugt, dass in Zukunft fortschrittlichere Deep-Learning-Architekturen für eine effektivere Spracherkennungssysteme entwickelt werden, die den hier diskutierten Netzwerken in vielerlei Hinsicht überlegen sind. Das Verständnis über die Struktur der Sprache, deren Dynamik und ihrer Repräsentation treiben den Forschungsfortschritt weiter voran. Ansätze, die in der Literatur zu finden sind, gehen davon aus, weitere Informationsquellen einzubeziehen, um die Qualität weiter zu verbessern. In diesem Zusammenhang wird in {\cite{Yu.2014}} das nutzen visueller Daten erwähnt. Dabei werden Merkmale aus interessanten Gesichtsregionen extrahiert. Da visuelle Informationen unabhängig von akustischem Rauschen sind, soll hier eine Verbesserung erzielt werden. Offen bleibt die Frage, welche Ansätze in Zukunft entwickelt werden, um die Interaktion mit Spracherkennungssystemen zu einem natürlichen Prozess zu machen. Da Maschinen die Welt nicht verstehen wie wir, ist es schwierig aus Tönen den gesamten Kontext zu verstehen. Weitere Forschungen können hier anknüpfen und sich mit potentiellen Möglichkeiten zur Lösung dieses Problems auseinandersetzen.  

% Can use something like this to put references on a page
% by themselves when using endfloat and the captionsoff option.
\ifCLASSOPTIONcaptionsoff
  \newpage
\fi
	
	\bibliographystyle{IEEEtran}
	\bibliography{references}
	
	% that's all folks
\end{document}