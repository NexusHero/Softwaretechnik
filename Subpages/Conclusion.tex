\section{Fazit}
Unter den beschriebenen Voraussetzungen – Verarbeitung kontinuierlicher Datenströme, Unterstützung mehrerer Sprachen und
zu wenige Trainingsdaten für einige Sprachen – sind die folgenden Verfahren am geeignetsten:

\begin{itemize}
  \item \textit{LSTM-Netze.} In den hidden layern werden durch die gemeinsame Nutzung von Phonemen mehrere Sprachen verarbeitet.
  \item \textit{ReLUs.} ??
  \item \textit{Multitasking Learning.} ??
\end{itemize}

Nachteile/Herausforderungen dieser Verfahren.
Verbesserungen mit maschinellem Lernen.
Höhere Genauigkeit bei multilingualer Spracherkennung als bei der monolingualen Spracherkennung.

\section{Ausblick}
- Emotionserkennung anhand multilingualer Spracherkennungssystemen? https://arxiv.org/abs/1803.00357
- Konversationen mit einem multilingualen Spracherkennungssystem: Google Duplex oder https://www.microsoft.com/en-us/research/wp-content/uploads/2017/08/ms\_swbd17-2.pdf

% Can use something like this to put references on a page
% by themselves when using endfloat and the captionsoff option.
\ifCLASSOPTIONcaptionsoff
  \newpage
\fi