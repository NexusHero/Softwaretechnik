\section{Diskussion und Ausblick}
Die menschliche Sprache ist der natürlichste Weg etwas zu kommunizieren, so ist es nicht verwunderlich, dass das Interesse an Deep learning-Ansätzen zur Spracherkennung und den damit verbundenen Anwendungen steigt. In dieser Arbeit wurde das Gegenstück zu den konventionellen, stochastischen Modellen beleuchtet - die Recurrent Neural Networks mit der Erweiterung der LSTM-Struktur. Dabei geht hervor, dass bei diesen Netzen die einzelnen Neuronen nicht isoliert betrachtet werden können. Vielmehr hängt deren Zustand und Aktivierung von den Aktivitäten anderer Neuronen bzw. Zellen ab. Vorhergehende Ereignissen beeinflussen den Zustand. Durch die dynamische Rekursion und den Gattern wird ein Gedächtnis geschaffen, mit welchem sich die Netze an vergangene Zustände erinnern können und aufgrund dieser Erfahrungen genauere Vorhersagen treffen. Vor allem bei Datensequenzen ist dies hilfreich. Da die menschliche Sprache lediglich eine Sequenz von Tönen ist, eignet sich diese Form von Netzwerken ideal.\newline
Multilinguale Spracherkennungssysteme haben mit verschiedenen Problemen zu kämpfen. Dazu zählt beispielsweise die Ressourcenknappheit einiger Sprachen, wodurch die Kernidee eines solchen Systems vom gemeinsamen Verwenden der Phoneme ausgeht. Um erlerntes Wissen gemeinsam zu nutzen, wird ein solches System über das Multitask Learning trainiert. Es ging hervor, dass mit der multilingualen Spracherkennung bessere Ergebnisse erzielt werden, als mit der monolingualen Erkennung. Dies ist auf das gemeinsame Nutzen der Phoneme zurückzuführen. Heutige Genauigkeiten beim Erkennen von Sprachen erreichen die Wortfehlerrate eines Menschen. Somit ist das reine Verstehen bald nicht mehr das hauptsächliche Problem. Die Autoren sind der Meinung, dass eine natürliche Interaktion mit einem Spracherkennungssystem dennoch schwierig bleibt, solange das System keine Kenntnisse über seine Umwelt hat. Beispielsweise klingen im Deutschen die Worte Meer und mehr gleich, haben jedoch nichts gemeinsam. Diese Homophone lassen sich zwar verstehen, das Spracherkennungssystem erkennt allerdings nicht den Kontext und es könnte zu einer inkorrekten Vorhersage führen. Es bestehen auch weitere, zahlreiche limitierende Faktoren. Das Mapping von Wörtern und Sequenzen aus Phonemen braucht Linguistikexperten und stellt eine Herausforderung dar. Schließlich müssen sämtliche Phoneme der Sprache identifiziert werden. Auch der Stil beim Sprechen verändert sich ständig und ist nie gleich zwischen unterschiedlichen Sprechern. Gesprochene Wörter beeinflussen die Betonung der nächsten Worte.\newline
Die Zukunft im Bereich des maschinellen Lernens bleibt spannend. Wir sind überzeugt, dass in Zukunft fortschrittlichere Deep Learning-Architekturen für effektivere Spracherkennungssysteme entwickelt werden, die den hier diskutierten Netzwerken in vielerlei Hinsicht überlegen sind. Das Verständnis über die Struktur der Sprache, deren Dynamik und ihrer Repräsentation treiben den Forschungsfortschritt weiter voran. Ansätze, die in der Literatur zu finden sind, gehen davon aus, weitere Informationsquellen einzubeziehen, um die Qualität weiter zu verbessern. In diesem Zusammenhang wird in {\cite{Yu.2014}} das nutzen visueller Daten erwähnt. Dabei werden Merkmale aus interessanten Gesichtsregionen extrahiert. Da visuelle Informationen unabhängig von akustischem Rauschen sind, soll hier eine Verbesserung erzielt werden. Offen bleibt die Frage, welche Ansätze in Zukunft entwickelt werden, um die Interaktion mit Spracherkennungssystemen zu einem natürlichen Prozess zu machen. Da Maschinen die Welt nicht verstehen wie wir, ist es schwierig aus Tönen den gesamten Kontext zu verstehen. Weitere Forschungen können hier anknüpfen und sich mit potentiellen Möglichkeiten zur Lösung dieses Problems auseinandersetzen.

% Can use something like this to put references on a page
% by themselves when using endfloat and the captionsoff option.
\ifCLASSOPTIONcaptionsoff
  \newpage
\fi